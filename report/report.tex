\documentclass[a4paper, 11pt, oneside]{article}

\usepackage[utf8]{inputenc}
\usepackage[T1]{fontenc}
\usepackage[english]{babel}
\usepackage{array}
\usepackage{shortvrb}
\usepackage{listings}
\usepackage[fleqn]{amsmath}
\usepackage{amsfonts}
\usepackage{fullpage}
\usepackage{enumerate}
\usepackage{enumitem}
\usepackage{graphicx}
\usepackage{subfigure}
\usepackage{alltt}
\usepackage{indentfirst}
\usepackage{eurosym}
\usepackage{listings}
\usepackage{titlesec, blindtext, color}
\usepackage{float}
\usepackage[colorlinks, linkcolor=blue]{hyperref}
\usepackage[nameinlink,noabbrev]{cleveref}

\usepackage{titling}
\renewcommand\maketitlehooka{\null\mbox{}\vfill}
\renewcommand\maketitlehookd{\vfill\null}

\definecolor{mygray}{rgb}{0.5,0.5,0.5}
\definecolor{pink1}{rgb}{0.858, 0.188, 0.478}
\definecolor{sienna}{rgb}{0.53, 0.18, 0.09}
\definecolor{sepia}{rgb}{0.44, 0.26, 0.08}
\definecolor{midnightblue}{rgb}{0.1, 0.1, 0.44}

\renewcommand{\lstlistingname}{Code}

\lstset{
    language=C, % Utilisation du langage C
    commentstyle={\color{MidnightBlue}}, % Couleur des commentaires
    frame=single, % Entoure le code d'un joli cadre
    rulecolor=\color{black}, % Couleur de la ligne qui forme le cadre
    numbers=left, % Ajoute une numérotation des lignes à gauche
    numbersep=5pt, % Distance entre les numérots de lignes et le code
    numberstyle=\tiny\color{mygray}, % Couleur des numéros de lignes
    basicstyle=\tt\footnotesize, 
    tabsize=3, % Largeur des tabulations par défaut
    extendedchars=true, 
    captionpos=b, % sets the caption-position to bottom
    texcl=true, % Commentaires sur une ligne interprétés en Latex
    showstringspaces=false, % Ne montre pas les espace dans les chaines de caractères
    escapeinside={(>}{<)}, % Permet de mettre du latex entre des <( et )>.
    inputencoding=utf8,
    literate=
  {á}{{\'a}}1 {é}{{\'e}}1 {í}{{\'i}}1 {ó}{{\'o}}1 {ú}{{\'u}}1
  {Á}{{\'A}}1 {É}{{\'E}}1 {Í}{{\'I}}1 {Ó}{{\'O}}1 {Ú}{{\'U}}1
  {à}{{\`a}}1 {è}{{\`e}}1 {ì}{{\`i}}1 {ò}{{\`o}}1 {ù}{{\`u}}1
  {À}{{\`A}}1 {È}{{\`E}}1 {Ì}{{\`I}}1 {Ò}{{\`O}}1 {Ù}{{\`U}}1
  {ä}{{\"a}}1 {ë}{{\"e}}1 {ï}{{\"i}}1 {ö}{{\"o}}1 {ü}{{\"u}}1
  {Ä}{{\"A}}1 {Ë}{{\"E}}1 {Ï}{{\"I}}1 {Ö}{{\"O}}1 {Ü}{{\"U}}1
  {â}{{\^a}}1 {ê}{{\^e}}1 {î}{{\^i}}1 {ô}{{\^o}}1 {û}{{\^u}}1
  {Â}{{\^A}}1 {Ê}{{\^E}}1 {Î}{{\^I}}1 {Ô}{{\^O}}1 {Û}{{\^U}}1
  {œ}{{\oe}}1 {Œ}{{\OE}}1 {æ}{{\ae}}1 {Æ}{{\AE}}1 {ß}{{\ss}}1
  {ű}{{\H{u}}}1 {Ű}{{\H{U}}}1 {ő}{{\H{o}}}1 {Ő}{{\H{O}}}1
  {ç}{{\c c}}1 {Ç}{{\c C}}1 {ø}{{\o}}1 {å}{{\r a}}1 {Å}{{\r A}}1
  {€}{{\euro}}1 {£}{{\pounds}}1 {«}{{\guillemotleft}}1
  {»}{{\guillemotright}}1 {ñ}{{\~n}}1 {Ñ}{{\~N}}1 {¿}{{?`}}1
}


\newcommand{\ClassName}{INFO-0940: Operating Systems}
\newcommand{\ProjectName}{Project 1: Tracer - Report}
\newcommand{\AcademicYear}{2020 - 2021}

%%%% Page de garde %%%%

\title{\ClassName\\\vspace*{0.8cm}\ProjectName\vspace{0.8cm}}
\author{Goffart Maxime \\180521 \and Joris Olivier \\ 182113}
\date{\vspace{1cm}Academic year \AcademicYear}

\begin{document}

%%% Page de garde %%%
\begin{titlingpage}
{\let\newpage\relax\maketitle}
\end{titlingpage}

%%%%%%%%%%%%%%%%%%%%%%%%%%%%%%%%%%%%%%%%%%%%%%

\section{Implementation}
\paragraph{}In this section, we will present our implementation. We will start with the structure of the code. Then, we will briefly explained the data structures that we are using.

\subsection{Code structure}
\paragraph{}Our implementation is composed of multiple source and header files. Our files are the following:
\begin{itemize}
	\item \texttt{tracer.c} The file which contains the main function of the tracer. Based on the given arguments, it starts the adequate mode (profiler or syscall).
	\item \texttt{syscall\{.c,.h\}} Represents the mode of the program that traces the system calls made by the tracee.
	\item \texttt{profiler\{.c,.h\}} Represents the mode of the program that traces the function calls made by the tracee.
	\item \texttt{file\_sys\_calls\{.c,.h\}} Represents the module that allows to load the file that contains the mapping between the ids and the names of the system calls.
	\item \texttt{functions\_addresses\{.c,.h\}} Represents the module that allows to load the mapping between the functions' names and their addresses. It requires \texttt{nm} and \texttt{objdump} to work.
	\item \texttt{load\_param\{.c,.h\}} Represents the module that allows to parse the arguments of the program.
\end{itemize}

\subsection{Data structures}
\paragraph{}Throughout the project, we used multiple data structures. Here is a list of the data structure that we are using \footnote{The explanations of the different fields are available in the source files}:
\begin{itemize}
	\item In \texttt{profiler\{.c,.h\}}, we have:
		\begin{itemize}
			\item \texttt{Profiler} represents the data of the profiler. After running \texttt{run\_profiler(char*)}, it contains the data of the profiling.
			\item \texttt{Func\_call} represents a function called by the tracee. An explanation of the different fields is available in the source file.
		\end{itemize}
	\item In \texttt{file\_sys\_calls\{.c,.h\}}, we have \texttt{FileSysCalls} which represents the mapping between the system calls' names and ids.
	\item In \texttt{functions\_addresses\{.c,.h\}}, we have:
		\begin{itemize}
			\item \texttt{FunctionAddresses} which represents, as a linked list of \texttt{Mapping}, the mapping between the functions' symbols and addresses.
			\item \texttt{Mapping} which represents, as a node of a linked list, a mapping between a function's symbol and address.
		\end{itemize}
\end{itemize}

\section{Specific questions}
\paragraph{}In this section, we will answer to the questions asked in the section 6 of the assignment.
\begin{enumerate}
	\item \textbf{What are the opcodes that you have used to detect the \texttt{call} and \texttt{ret} insturctions?}\\
	To detect the \texttt{call} instruction, we used the opcode: 0xe8. We are only using this particular opcode because we are only considering the relative calls.\\
	To detect the \texttt{ret} instruction, we used the opcodes: 0xc2, 0xc3, 0xca, 0xcb, 0xf3c3, and 0xf2c3.
	\item \textbf{Why are there many functions called before the "real" program and what is/are their purpose(s)?}\\
	Their purposes are to set everything in order for the process to be runnable (call to \_\_libc\_start\_main) including setting up the thread inside the process (call to\\\_\_pthread\_initialize\_minimal).
	\item \textbf{According to you, what is the reason for statically compiling the "tracee" program?}\\
	In order to get the functions' names (including the ones from libraries), we are using \texttt{nm} and \texttt{objdump}. These two programs rely on the ELF file generated by the compilation process\footnote{Here, by compilation process, we mean preprocessing, compiling, assembly, \textbf{and} linking.}. Yet, dynamic libraries are loaded at runtime where static libraries are integrated in the executable during the linking phase. Because we need the functions from libraries to be in the executable, due to the usage of \texttt{nm} and \texttt{objdump}, it forces us to use static libraries, which themselves implies static linking.
\end{enumerate}

\section{Estimation of spent time}
\paragraph{}Unfortunately, we were working, at the same time, on other projects, so we do not have a clear idea of how much time we spent on this project in particular.


\end{document}